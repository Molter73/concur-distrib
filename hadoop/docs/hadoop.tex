\documentclass[a4paper]{article}

\usepackage{titling}
\usepackage{fullpage}
\usepackage{fontenc}
\usepackage{listings}
\usepackage{mathptmx}
\usepackage{hyperref}
\usepackage{tikz}
\usepackage{graphicx}
\usepackage{float}
\usepackage{parskip}
\usepackage[none]{hyphenat}

\renewcommand{\contentsname}{Contenidos}

\hypersetup{
    colorlinks=true,
    linkcolor=blue,
    filecolor=magenta,
    urlcolor=blue,
}

\urlstyle{same}

\begin{document}

\title{Hadoop}
\author{Moltrasio, Mauro Ezequiel}
\date{}
\renewcommand{\abstractname}{\vspace{-\baselineskip}}

\begin{titlingpage}
    \maketitle
    \begin{abstract}

        Clase: Programación concurrente y distribuida.

        Actividad 3: Hadoop.

        Profesor: José Delgado Pérez.
    \end{abstract}
\end{titlingpage}

\maketitle
\tableofcontents

\section{Introducción}

En este documento se detallarán los pasos tomados para realizar la
implementación de una aplicación MapReduce para contar severidades de logs en
un archivo. Para esto se utilizó la imagen cloudera/quickstart:latest
como base y se la modificó para ejecutar nuestra aplicación.

\section{Implementación del código}

Como se adelantaba en la introducción, se utilizó la imagen
cloudera/quickstart:latest que provee un proyecto esqueleto que se
puede modificar para cumplir con nuestro objetivo. Se tomaron las clases
StubDriver, StubMapper y StubReducer, renombrándolas reemplazando Stub por
LogCounter y se procedió a modificarlas.

Primero se modificó la clase LogCounterMapper para que utilice un regex,
capturando la severidad del log y emitiendo un par clave-valor donde:

\begin{itemize}
    \item Clave es el valor de la severidad extraída por el regex.
    \item Valor siempre es igual a 1.
\end{itemize}

La clase LogCounterReducer simplemente tomará la salida de LogCounterMapper y
sumará los valores. Al ser que los valores siempre son 1, el resultado final
será la cuenta de ocurrencias de cada severidad en el log.

Finalmente, la clase LogCounterDriver realiza toda la configuración necesaria
para ejecutar el trabajo MapReduce y emitir el resultado.

\section{Modificación de la imagen del contenedor}

Para simplificar la ejecución del código y garantizar que la compilación del
mismo sea consistente, se creó un archivo Dockerfile con dos etapas. La primera
realiza la compilación de las clases y crea un jar. La segunda modifica la
imagen cloudera/quickstart:latest con el jar generado y agregando
un script entrypoint.sh encargado de iniciar el ambiente y ejecutar el trabajo
MapReduce. Para crear dicha imagen basta con ejecutar el siguiente comando en
la raíz del proyecto:

\begin{lstlisting}[
    language=bash,
    basicstyle=\small
]
 $ docker build .
\end{lstlisting}

La salida de dicho comando es la siguiente:

\begin{lstlisting}[
    language=bash,
    basicstyle=\small
]
[1/2] STEP 1/5: FROM cloudera/quickstart:latest AS builder
[1/2] STEP 2/5: COPY src/ src/
--> c0a02d8eea25
[1/2] STEP 3/5: RUN mkdir -p target/
--> bf153e48b178
[1/2] STEP 4/5: RUN javac -cp /usr/lib/hadoop/*:/usr/lib/hadoop-mapreduce/* src/*.java -d target/ -Xlint
warning: [path] bad path element "/usr/lib/hadoop-mapreduce/jaxb-api.jar": no such file or directory
warning: [path] bad path element "/usr/lib/hadoop-mapreduce/activation.jar": no such file or directory
warning: [path] bad path element "/usr/lib/hadoop-mapreduce/jsr173_1.0_api.jar": no such file or directory
warning: [path] bad path element "/usr/lib/hadoop-mapreduce/jaxb1-impl.jar": no such file or directory
4 warnings
--> dcff47880e50
[1/2] STEP 5/5: RUN jar -cvf logcounter.jar -C target/ .
added manifest
adding: LogCounterDriver.class(in = 1707) (out= 863)(deflated 49%)
adding: LogCounterMapper.class(in = 2005) (out= 851)(deflated 57%)
adding: LogCounterReducer.class(in = 1604) (out= 669)(deflated 58%)
--> 90bc1588b9c4
[2/2] STEP 1/7: FROM cloudera/quickstart:latest
[2/2] STEP 2/7: RUN mkdir -p /workspace/input
--> 8bc88ef1dea5
[2/2] STEP 3/7: RUN mkdir -p /workspace/output
--> a4b682ed1b47
[2/2] STEP 4/7: WORKDIR /workspace
--> eab8cd216cf0
[2/2] STEP 5/7: COPY --from=builder logcounter.jar .
--> 20a79ef8ef68
[2/2] STEP 6/7: COPY scripts/entrypoint.sh /usr/local/bin/entrypoint.sh
--> 2de06451a276
[2/2] STEP 7/7: ENTRYPOINT entrypoint.sh
[2/2] COMMIT quay.io/mmoltras/cloudera
--> a8e1ad4ec23c
Successfully tagged quay.io/mmoltras/cloudera:latest
a8e1ad4ec23c17b405cb082fb63f18a28ffdc9b5b6a29620394acd12c9f734df
\end{lstlisting}

La imagen también se ha publicado en un registro público bajo el tag:
quay.io/mmoltras/cloudera:latest

\section{Ejecución del contenedor}

Al no tener acceso a un archivo de log con la salida querida, se creó un script
log-generator.py que se puede utilizar para generar dicho archivo. Una vez
creado el archivo de log, basta con montar el directorio que lo contiene en
el contenedor bajo la ruta /workspace/input y el script entrypoint.sh se
encargará de inicializar el ambiente y ejecutar el trabajo MapReduce.

El comando para ejecutar el contenedor quedaría como se ve a continuación:

\begin{lstlisting}[
    language=bash,
    basicstyle=\small
]
 $ docker run --rm \
    --hostname=quickstart.cloudera \
    --privileged \
    -v /path/to/dir/holding/log:/workspace/input \
    quay.io/mmoltras/cloudera
\end{lstlisting}

Una vez termina la ejecución se tendrá la cantidad de severidades que se
encontraron en la terminal:

\begin{lstlisting}[
    language=bash,
    basicstyle=\small
]
+ hadoop fs -cat '/workspace/output/*'
INFO    31
SEVERE  36
WARN    33
\end{lstlisting}

La salida completa del contenedor es muy extensa, por lo que se agregará en un
anexo a parte.

\section{Repositorio}

El repositorio contendrá el código de todas las tareas de la materia en
subdirectorios. Se decidió este enfoque en lugar de un repositorio por proyecto
porque mi cuenta ya tiene demasiados repositorios.

https://github.com/Molter73/concur-distrib/tree/main/hadoop

\appendix
\section{Salida de ejecución del contenedor}
\begin{lstlisting}[
    language=bash,
    basicstyle=\small
]
+ DAEMONS='    mysqld     cloudera-quickstart-init     zookeeper-server     hadoop-hdfs-datanode     hadoop-hdfs-journalnode     hadoop-hdfs-namenode     hadoop-hdfs-secondarynamenode     hadoop-httpfs     hadoop-mapreduce-historyserver     hadoop-yarn-nodemanager     hadoop-yarn-resourcemanager     hbase-master     hbase-rest     hbase-thrift     hive-metastore     hive-server2     sqoop2-server     spark-history-server     hbase-regionserver     impala-state-store     oozie     solr-server     impala-catalog     impala-server'
+ for daemon in '${DAEMONS}'
+ sudo service mysqld start
Starting mysqld:  [  OK  ]
+ for daemon in '${DAEMONS}'
+ sudo service cloudera-quickstart-init start

if [ "$1" == "start" ] ; then
    if [ "${EC2}" == 'true' ]; then
        FIRST_BOOT_FLAG=/var/lib/cloudera-quickstart/.ec2-key-installed
        if [ ! -f "${FIRST_BOOT_FLAG}" ]; then
            METADATA_API=http://169.254.169.254/latest/meta-data
            KEY_URL=${METADATA_API}/public-keys/0/openssh-key
            SSH_DIR=/home/cloudera/.ssh
            mkdir -p ${SSH_DIR}
            chown cloudera:cloudera ${SSH_DIR}
            curl ${KEY_URL} >> ${SSH_DIR}/authorized_keys
            touch ${FIRST_BOOT_FLAG}
        fi
    fi
    if [ "${DOCKER}" != 'true' ]; then
        if [ -f /sys/kernel/mm/redhat_transparent_hugepage/defrag ]; then
            echo never > /sys/kernel/mm/redhat_transparent_hugepage/defrag
        fi

        cloudera-quickstart-ip
        HOSTNAME=quickstart.cloudera
        hostname ${HOSTNAME}
        sed -i -e "s/HOSTNAME=.*/HOSTNAME=${HOSTNAME}/" /etc/sysconfig/network
    fi

    (
        cd /var/lib/cloudera-quickstart/tutorial;
        nohup python -m SimpleHTTPServer 80 &
    )

    # TODO: check for expired CM license and update config.js accordingly
fi
+ '[' start == start ']'
+ '[' '' == true ']'
+ '[' true '!=' true ']'
+ cd /var/lib/cloudera-quickstart/tutorial
+ nohup python -m SimpleHTTPServer 80

+ for daemon in '${DAEMONS}'
+ sudo service zookeeper-server start
JMX enabled by default
Using config: /etc/zookeeper/conf/zoo.cfg
Starting zookeeper ... STARTED
+ for daemon in '${DAEMONS}'
+ sudo service hadoop-hdfs-datanode start
starting datanode, logging to /var/log/hadoop-hdfs/hadoop-hdfs-datanode-quickstart.cloudera.out
Started Hadoop datanode (hadoop-hdfs-datanode):[  OK  ]
+ for daemon in '${DAEMONS}'
+ sudo service hadoop-hdfs-journalnode start
starting journalnode, logging to /var/log/hadoop-hdfs/hadoop-hdfs-journalnode-quickstart.cloudera.out
Started Hadoop journalnode:[  OK  ]
+ for daemon in '${DAEMONS}'
+ sudo service hadoop-hdfs-namenode start
starting namenode, logging to /var/log/hadoop-hdfs/hadoop-hdfs-namenode-quickstart.cloudera.out
Started Hadoop namenode:[  OK  ]
+ for daemon in '${DAEMONS}'
+ sudo service hadoop-hdfs-secondarynamenode start
starting secondarynamenode, logging to /var/log/hadoop-hdfs/hadoop-hdfs-secondarynamenode-quickstart.cloudera.out
Started Hadoop secondarynamenode:[  OK  ]
+ for daemon in '${DAEMONS}'
+ sudo service hadoop-httpfs start

Setting HTTPFS_HOME:          /usr/lib/hadoop-httpfs
Using   HTTPFS_CONFIG:        /etc/hadoop-httpfs/conf
Sourcing:                    /etc/hadoop-httpfs/conf/httpfs-env.sh
Using   HTTPFS_LOG:           /var/log/hadoop-httpfs/
Using   HTTPFS_TEMP:           /var/run/hadoop-httpfs
Setting HTTPFS_HTTP_PORT:     14000
Setting HTTPFS_ADMIN_PORT:     14001
Setting HTTPFS_HTTP_HOSTNAME: quickstart.cloudera
Setting HTTPFS_SSL_ENABLED: false
Setting HTTPFS_SSL_KEYSTORE_FILE:     /var/lib/hadoop-httpfs/.keystore
Setting HTTPFS_SSL_KEYSTORE_PASS:     password
Using   CATALINA_BASE:       /var/lib/hadoop-httpfs/tomcat-deployment
Using   HTTPFS_CATALINA_HOME:       /usr/lib/bigtop-tomcat
Setting CATALINA_OUT:        /var/log/hadoop-httpfs//httpfs-catalina.out
Using   CATALINA_PID:        /var/run/hadoop-httpfs/hadoop-httpfs-httpfs.pid

Using   CATALINA_OPTS:
Adding to CATALINA_OPTS:     -Dhttpfs.home.dir=/usr/lib/hadoop-httpfs -Dhttpfs.config.dir=/etc/hadoop-httpfs/conf -Dhttpfs.log.dir=/var/log/hadoop-httpfs/ -Dhttpfs.temp.dir=/var/run/hadoop-httpfs -Dhttpfs.admin.port=14001 -Dhttpfs.http.port=14000 -Dhttpfs.http.hostname=quickstart.cloudera
Started Hadoop httpfs (hadoop-httpfs):[  OK  ]
+ for daemon in '${DAEMONS}'
+ sudo service hadoop-mapreduce-historyserver start
starting historyserver, logging to /var/log/hadoop-mapreduce/mapred-mapred-historyserver-quickstart.cloudera.out
23/12/13 17:42:28 INFO hs.JobHistoryServer: STARTUP_MSG:
/************************************************************
STARTUP_MSG: Starting JobHistoryServer
STARTUP_MSG:   host = quickstart.cloudera/10.88.0.7
STARTUP_MSG:   args = []
STARTUP_MSG:   version = 2.6.0-cdh5.7.0
STARTUP_MSG:   classpath = ...
STARTUP_MSG:   build = http://github.com/cloudera/hadoop -r c00978c67b0d3fe9f3b896b5030741bd40bf541a; compiled by 'jenkins' on 2016-03-23T18:36Z
STARTUP_MSG:   java = 1.7.0_67
************************************************************/
Started Hadoop historyserver:[  OK  ]
+ for daemon in '${DAEMONS}'
+ sudo service hadoop-yarn-nodemanager start
starting nodemanager, logging to /var/log/hadoop-yarn/yarn-yarn-nodemanager-quickstart.cloudera.out
Started Hadoop nodemanager:[  OK  ]
+ for daemon in '${DAEMONS}'
+ sudo service hadoop-yarn-resourcemanager start
starting resourcemanager, logging to /var/log/hadoop-yarn/yarn-yarn-resourcemanager-quickstart.cloudera.out
Started Hadoop resourcemanager:[  OK  ]
+ for daemon in '${DAEMONS}'
+ sudo service hbase-master start
starting master, logging to /var/log/hbase/hbase-hbase-master-quickstart.cloudera.out
Started HBase master daemon (hbase-master):[  OK  ]
+ for daemon in '${DAEMONS}'
+ sudo service hbase-rest start
starting rest, logging to /var/log/hbase/hbase-hbase-rest-quickstart.cloudera.out
Started HBase rest daemon (hbase-rest):[  OK  ]
+ for daemon in '${DAEMONS}'
+ sudo service hbase-thrift start
starting thrift, logging to /var/log/hbase/hbase-hbase-thrift-quickstart.cloudera.out
Started HBase thrift daemon (hbase-thrift):[  OK  ]
+ for daemon in '${DAEMONS}'
+ sudo service hive-metastore start
Starting Hive Metastore (hive-metastore):[  OK  ]
+ for daemon in '${DAEMONS}'
+ sudo service hive-server2 start
Started Hive Server2 (hive-server2):[  OK  ]
+ for daemon in '${DAEMONS}'
+ sudo service sqoop2-server start
Starting Sqoop Server:[  OK  ]
Sqoop home directory: /usr/lib/sqoop2
Setting SQOOP_HTTP_PORT:     12000
Setting SQOOP_ADMIN_PORT:     12001
Using   CATALINA_OPTS:       -Xmx1024m
Adding to CATALINA_OPTS:    -Dsqoop.http.port=12000 -Dsqoop.admin.port=12001
+ for daemon in '${DAEMONS}'
+ sudo service spark-history-server start
Starting Spark history-server (spark-history-server):[  OK  ]
+ for daemon in '${DAEMONS}'
+ sudo service hbase-regionserver start
Starting Hadoop HBase regionserver daemon: starting regionserver, logging to /var/log/hbase/hbase-hbase-regionserver-quickstart.cloudera.out
hbase-regionserver.
+ for daemon in '${DAEMONS}'
+ sudo service impala-state-store start
Started Impala State Store Server (statestored):[  OK  ]
+ for daemon in '${DAEMONS}'
+ sudo service oozie start

Setting OOZIE_HOME:          /usr/lib/oozie
Sourcing:                    /usr/lib/oozie/bin/oozie-env.sh
  setting JAVA_LIBRARY_PATH="$JAVA_LIBRARY_PATH:/usr/lib/hadoop/lib/native"
  setting OOZIE_DATA=/var/lib/oozie
  setting OOZIE_CATALINA_HOME=/usr/lib/bigtop-tomcat
  setting CATALINA_TMPDIR=/var/lib/oozie
  setting CATALINA_PID=/var/run/oozie/oozie.pid
  setting CATALINA_BASE=/var/lib/oozie/tomcat-deployment
  setting OOZIE_HTTPS_PORT=11443
  setting OOZIE_HTTPS_KEYSTORE_PASS=password
  setting CATALINA_OPTS="$CATALINA_OPTS -Doozie.https.port=${OOZIE_HTTPS_PORT}"
  setting CATALINA_OPTS="$CATALINA_OPTS -Doozie.https.keystore.pass=${OOZIE_HTTPS_KEYSTORE_PASS}"
  setting CATALINA_OPTS="$CATALINA_OPTS -Xmx1024m"
  setting OOZIE_CONFIG=/etc/oozie/conf
  setting OOZIE_LOG=/var/log/oozie
Using   OOZIE_CONFIG:        /etc/oozie/conf
Sourcing:                    /etc/oozie/conf/oozie-env.sh
  setting JAVA_LIBRARY_PATH="$JAVA_LIBRARY_PATH:/usr/lib/hadoop/lib/native"
  setting OOZIE_DATA=/var/lib/oozie
  setting OOZIE_CATALINA_HOME=/usr/lib/bigtop-tomcat
  setting CATALINA_TMPDIR=/var/lib/oozie
  setting CATALINA_PID=/var/run/oozie/oozie.pid
  setting CATALINA_BASE=/var/lib/oozie/tomcat-deployment
  setting OOZIE_HTTPS_PORT=11443
  setting OOZIE_HTTPS_KEYSTORE_PASS=password
  setting CATALINA_OPTS="$CATALINA_OPTS -Doozie.https.port=${OOZIE_HTTPS_PORT}"
  setting CATALINA_OPTS="$CATALINA_OPTS -Doozie.https.keystore.pass=${OOZIE_HTTPS_KEYSTORE_PASS}"
  setting CATALINA_OPTS="$CATALINA_OPTS -Xmx1024m"
  setting OOZIE_CONFIG=/etc/oozie/conf
  setting OOZIE_LOG=/var/log/oozie
Setting OOZIE_CONFIG_FILE:   oozie-site.xml
Using   OOZIE_DATA:          /var/lib/oozie
Using   OOZIE_LOG:           /var/log/oozie
Setting OOZIE_LOG4J_FILE:    oozie-log4j.properties
Setting OOZIE_LOG4J_RELOAD:  10
Setting OOZIE_HTTP_HOSTNAME: quickstart.cloudera
Setting OOZIE_HTTP_PORT:     11000
Setting OOZIE_ADMIN_PORT:     11001
Using   OOZIE_HTTPS_PORT:     11443
Setting OOZIE_BASE_URL:      http://quickstart.cloudera:11000/oozie
Using   CATALINA_BASE:       /var/lib/oozie/tomcat-deployment
Setting OOZIE_HTTPS_KEYSTORE_FILE:     /var/lib/oozie/.keystore
Using   OOZIE_HTTPS_KEYSTORE_PASS:     password
Setting OOZIE_INSTANCE_ID:       quickstart.cloudera
Setting CATALINA_OUT:        /var/log/oozie/catalina.out
Using   CATALINA_PID:        /var/run/oozie/oozie.pid

Using   CATALINA_OPTS:        -Doozie.https.port=11443 -Doozie.https.keystore.pass=password -Xmx1024m -Doozie.https.port=11443 -Doozie.https.keystore.pass=password -Xmx1024m -Dderby.stream.error.file=/var/log/oozie/derby.log
Adding to CATALINA_OPTS:     -Doozie.home.dir=/usr/lib/oozie -Doozie.config.dir=/etc/oozie/conf -Doozie.log.dir=/var/log/oozie -Doozie.data.dir=/var/lib/oozie -Doozie.instance.id=quickstart.cloudera -Doozie.config.file=oozie-site.xml -Doozie.log4j.file=oozie-log4j.properties -Doozie.log4j.reload=10 -Doozie.http.hostname=quickstart.cloudera -Doozie.admin.port=11001 -Doozie.http.port=11000 -Doozie.https.port=11443 -Doozie.base.url=http://quickstart.cloudera:11000/oozie -Doozie.https.keystore.file=/var/lib/oozie/.keystore -Doozie.https.keystore.pass=password -Djava.library.path=:/usr/lib/hadoop/lib/native:/usr/lib/hadoop/lib/native

+ for daemon in '${DAEMONS}'
+ sudo service solr-server start
Starting Solr server daemon:[  OK  ]
+ for daemon in '${DAEMONS}'
+ sudo service impala-catalog start
Started Impala Catalog Server (catalogd) :[  OK  ]
+ for daemon in '${DAEMONS}'
+ sudo service impala-server start
Started Impala Server (impalad):[  OK  ]
+ hadoop fs -mkdir -p /workspace/input
+ hadoop fs -put /workspace/input/test.log /workspace/input
+ hadoop jar /workspace/logcounter.jar LogCounterDriver /workspace/input /workspace/output
23/12/13 17:43:26 INFO client.RMProxy: Connecting to ResourceManager at /0.0.0.0:8032
23/12/13 17:43:27 INFO input.FileInputFormat: Total input paths to process : 1
23/12/13 17:43:27 INFO mapreduce.JobSubmitter: number of splits:1
23/12/13 17:43:27 INFO mapreduce.JobSubmitter: Submitting tokens for job: job_1702489362687_0001
23/12/13 17:43:28 INFO impl.YarnClientImpl: Submitted application application_1702489362687_0001
23/12/13 17:43:28 INFO mapreduce.Job: The url to track the job: http://quickstart.cloudera:8088/proxy/application_1702489362687_0001/
23/12/13 17:43:28 INFO mapreduce.Job: Running job: job_1702489362687_0001
23/12/13 17:43:43 INFO mapreduce.Job: Job job_1702489362687_0001 running in uber mode : false
23/12/13 17:43:43 INFO mapreduce.Job:  map 0% reduce 0%
23/12/13 17:43:54 INFO mapreduce.Job:  map 100% reduce 0%
23/12/13 17:44:00 INFO mapreduce.Job:  map 100% reduce 100%
23/12/13 17:44:00 INFO mapreduce.Job: Job job_1702489362687_0001 completed successfully
23/12/13 17:44:00 INFO mapreduce.Job: Counters: 49
        File System Counters
                FILE: Number of bytes read=1180
                FILE: Number of bytes written=229353
                FILE: Number of read operations=0
                FILE: Number of large read operations=0
                FILE: Number of write operations=0
                HDFS: Number of bytes read=3795
                HDFS: Number of bytes written=26
                HDFS: Number of read operations=6
                HDFS: Number of large read operations=0
                HDFS: Number of write operations=2
        Job Counters
                Launched map tasks=1
                Launched reduce tasks=1
                Data-local map tasks=1
                Total time spent by all maps in occupied slots (ms)=7762
                Total time spent by all reduces in occupied slots (ms)=3589
                Total time spent by all map tasks (ms)=7762
                Total time spent by all reduce tasks (ms)=3589
                Total vcore-seconds taken by all map tasks=7762
                Total vcore-seconds taken by all reduce tasks=3589
                Total megabyte-seconds taken by all map tasks=7948288
                Total megabyte-seconds taken by all reduce tasks=3675136
        Map-Reduce Framework
                Map input records=100
                Map output records=100
                Map output bytes=974
                Map output materialized bytes=1180
                Input split bytes=121
                Combine input records=0
                Combine output records=0
                Reduce input groups=3
                Reduce shuffle bytes=1180
                Reduce input records=100
                Reduce output records=3
                Spilled Records=200
                Shuffled Maps =1
                Failed Shuffles=0
                Merged Map outputs=1
                GC time elapsed (ms)=90
                CPU time spent (ms)=1590
                Physical memory (bytes) snapshot=555298816
                Virtual memory (bytes) snapshot=2749214720
                Total committed heap usage (bytes)=754974720
        Shuffle Errors
                BAD_ID=0
                CONNECTION=0
                IO_ERROR=0
                WRONG_LENGTH=0
                WRONG_MAP=0
                WRONG_REDUCE=0
        File Input Format Counters
                Bytes Read=3674
        File Output Format Counters
                Bytes Written=26
+ hadoop fs -cat '/workspace/output/*'
INFO    38
SEVERE  37
WARN    25
\end{lstlisting}

\end{document}
