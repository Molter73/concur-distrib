\documentclass[a4paper]{article}

\usepackage{titling}
\usepackage{fullpage}
\usepackage{fontenc}
\usepackage{mathptmx}
\usepackage{hyperref}
\usepackage{tikz}
\usepackage{graphicx}
\usepackage{float}
\usepackage{parskip}

\renewcommand{\contentsname}{Contenidos}

\hypersetup{
    colorlinks=true,
    linkcolor=blue,
    filecolor=magenta,
    urlcolor=blue,
}

\urlstyle{same}

\begin{document}

\title{Semáforos y monitores}
\author{Moltrasio, Mauro Ezequiel}
\date{}
\renewcommand{\abstractname}{\vspace{-\baselineskip}}

\begin{titlingpage}
    \maketitle
    \begin{abstract}

        Clase: Programación concurrente y distribuida.

        Actividad 2: Semáforos y monitores.

        Profesor: José Delgado Pérez.
    \end{abstract}
\end{titlingpage}

\maketitle
\tableofcontents

\section{Introducción}

En este documento se detallará la implementación de un proyecto en dos partes.
La primera parte es un manejador de recurso multi-hilo basado en un semáforo
binario, y la segunda es el manejo de acceso a un recurso compartido a través
de un monitor.

La implementación de la solución es un proyecto maven con dos submódulos.
El proyecto utiliza Java 1.8, las versiones espcíficas utilizadas son:

\begin{itemize}
    \item java version "1.8.0\_361"
    \item Java(TM) SE Runtime Environment (build 1.8.0\_361-b09)
\end{itemize}

\section{Semáforo binario}

En esta parte de la actividad se implementó un control de acceso multi-hilo a
un recurso ficticio mediante el uso de un semáforo binario. La solución cuenta
con una dos clases:
\begin{itemize}
    \item Process: Lanza una nueva hebra e intenta reservar y liberar recursos
    \item Resource: Controla el acceso a un recurso compartido.
\end{itemize}

De esta manera, la clase App que contiene el método main crea 4 hebras de la
clase Process y estas comienzan a reservar y liberar recursos llamando a los
métodos reserve y free de la clase estática Resource. Estos métodos a su vez
utilizan un semáforo binario para asegurar que la asignación de recursos no
posee condiciones de carrera.

\section{Monitor}

En esta actividad se simuló el acceso a un puente. Para esto se creó una clase
BridgeMonitor que utiliza un monitor para sincronizar el acceso al mismo. Este
monitor tiene dos métodos:

\begin{itemize}
    \item getOn: Toma el sentido en el que un auto intenta acceder al puente
        y decide si esta hebra debe esperar o está habilitada a subir al puente.
    \item getOff: Una vez se termina de cruzar el puente, este método retira
        el automóvil del mismo. Una vez se hayan bajado todos los automóviles,
        se marca el puente como vacío y se notifica a las hebras que puedan
        estar esperando para que comiencen su cruce.
\end{itemize}

La clase Car simplemente toma un monitor al crearse, decide un sentido de
circulación de forma aleatoria (sur-norte o norte-sur) y lanza una hebra que
intenta cruzar el puente.

Finalmente, la clase App es el punto de entrada principal, crea 100 autos a la
vez y estos realizan el cruce de forma ordenada gracias al monitor.

\section{Lecciones aprendidas}

Llevo varios años trabajando en entornos multi-hilo en distintos lenguajes, por
lo que el concepto de semáforo binario no es nuevo para mi. Sin embargo, al no
estar tan familiarizado con Java, es la primera vez que trabajo con monitores,
es un concepto interesante el tener métodos "auto-sincronizados", pero a fines
prácticos no lo veo tan diferente a utilizar mutex de forma explícita.

\section{Repositorio}

El repositorio contendrá el código de todas las tareas de la materia en
subdirectorios. Se decidió este enfoque en lugar de un repositorio por proyecto
porque mi cuenta ya tiene demasiados repositorios.

https://github.com/Molter73/concur-distrib/tree/main/semaphore-and-monitor

\end{document}
