\documentclass[a4paper]{article}

\usepackage{titling}
\usepackage{fullpage}
\usepackage{fontenc}
\usepackage{mathptmx}
\usepackage{hyperref}
\usepackage{tikz}
\usepackage{graphicx}
\usepackage{float}
\usepackage{parskip}

\renewcommand{\contentsname}{Contenidos}

\hypersetup{
    colorlinks=true,
    linkcolor=blue,
    filecolor=magenta,
    urlcolor=blue,
}

\urlstyle{same}

\begin{document}

\title{Hilos y sockets}
\author{Moltrasio, Mauro Ezequiel}
\date{}
\renewcommand{\abstractname}{\vspace{-\baselineskip}}

\begin{titlingpage}
    \maketitle
    \begin{abstract}

        Clase: Programación concurrente y distribuida.

        Actividad 1: Hilos y socket.

        Profesor: José Delgado Pérez.
    \end{abstract}
\end{titlingpage}

\maketitle
\tableofcontents

\section{Introducción}

Este documento detalla la implementación de un servidor que maneja conexiones
concurrentes de clientes simples.

\section{Implementación}

La implementación de la solución es un proyecto maven con dos submódulos, un
cliente y un servidor.

\subsection{Cliente}

El cliente consiste de una sola clase, encargada de realizar la conexión al
servidor, enviar los mensajes ingresados por el usuario y mostrar por pantalla
las respuestas del servidor.

\subsection{Servidor}

El servidor abre un socket y se mantiene escuchando por conexiones. Cada vez
que un cliente realiza una conexión, el servidor crea un nuevo hilo trabajador
(Worker) y este será el encargado de manejar toda la comunicación con ese
cliente.

\subsection{Worker}

Esta clase implementa la interfaz Runnable y es la encargada de manejar
la comunicación con los clientes. De esta manera, se conseguirá que se aíslen
las comunicaciones y se simplifica la codificación, ya que no debemos mantener
un registro del estado de cada cliente, el worker está dedicado y es
independiente del resto de conexiones.

Para ofrecer una serie de tareas sencillas a los clientes, se creó una
interface Activity que el worker utiliza para realizar un manejo generico
del estado. Se implementaron 4 actividades con las que el cliente puede
interactuar, la quinta opción requerida es cerrar la comunicación.

\begin{figure}[H]
    \centering
    \scalebox{.6}{
        \input{build/uml.tex}
    }
\end{figure}

\section{Lecciones aprendidas}

Llevo varios años trabajando en entornos multi-hilo, por lo que no puedo
afirmar haber aprendido demasiado más allá de los mecanismos específicos de
Java para la creación de hilos. Sin embargo, siempre es bueno volver a las
bases y repasar.

\section{Repositorio}

El repositorio contendrá el código de todas las tareas de la materia en
subdirectorios. Se decidió este enfoque en lugar de un repositorio por proyecto
porque mi cuenta ya tiene demasiados repositorios.

https://github.com/Molter73/concur-distrib/tree/main/threads-sockets

\end{document}
